
% Default to the notebook output style

    


% Inherit from the specified cell style.




    
\documentclass[11pt]{article}

    
    
    \usepackage[T1]{fontenc}
    % Nicer default font (+ math font) than Computer Modern for most use cases
    \usepackage{mathpazo}

    % Basic figure setup, for now with no caption control since it's done
    % automatically by Pandoc (which extracts ![](path) syntax from Markdown).
    \usepackage{graphicx}
    % We will generate all images so they have a width \maxwidth. This means
    % that they will get their normal width if they fit onto the page, but
    % are scaled down if they would overflow the margins.
    \makeatletter
    \def\maxwidth{\ifdim\Gin@nat@width>\linewidth\linewidth
    \else\Gin@nat@width\fi}
    \makeatother
    \let\Oldincludegraphics\includegraphics
    % Set max figure width to be 80% of text width, for now hardcoded.
    \renewcommand{\includegraphics}[1]{\Oldincludegraphics[width=.8\maxwidth]{#1}}
    % Ensure that by default, figures have no caption (until we provide a
    % proper Figure object with a Caption API and a way to capture that
    % in the conversion process - todo).
    \usepackage{caption}
    \DeclareCaptionLabelFormat{nolabel}{}
    \captionsetup{labelformat=nolabel}

    \usepackage{adjustbox} % Used to constrain images to a maximum size 
    \usepackage{xcolor} % Allow colors to be defined
    \usepackage{enumerate} % Needed for markdown enumerations to work
    \usepackage{geometry} % Used to adjust the document margins
    \usepackage{amsmath} % Equations
    \usepackage{amssymb} % Equations
    \usepackage{textcomp} % defines textquotesingle
    % Hack from http://tex.stackexchange.com/a/47451/13684:
    \AtBeginDocument{%
        \def\PYZsq{\textquotesingle}% Upright quotes in Pygmentized code
    }
    \usepackage{upquote} % Upright quotes for verbatim code
    \usepackage{eurosym} % defines \euro
    \usepackage[mathletters]{ucs} % Extended unicode (utf-8) support
    \usepackage[utf8x]{inputenc} % Allow utf-8 characters in the tex document
    \usepackage{fancyvrb} % verbatim replacement that allows latex
    \usepackage{grffile} % extends the file name processing of package graphics 
                         % to support a larger range 
    % The hyperref package gives us a pdf with properly built
    % internal navigation ('pdf bookmarks' for the table of contents,
    % internal cross-reference links, web links for URLs, etc.)
    \usepackage{hyperref}
    \usepackage{longtable} % longtable support required by pandoc >1.10
    \usepackage{booktabs}  % table support for pandoc > 1.12.2
    \usepackage[inline]{enumitem} % IRkernel/repr support (it uses the enumerate* environment)
    \usepackage[normalem]{ulem} % ulem is needed to support strikethroughs (\sout)
                                % normalem makes italics be italics, not underlines
    

    
    
    % Colors for the hyperref package
    \definecolor{urlcolor}{rgb}{0,.145,.698}
    \definecolor{linkcolor}{rgb}{.71,0.21,0.01}
    \definecolor{citecolor}{rgb}{.12,.54,.11}

    % ANSI colors
    \definecolor{ansi-black}{HTML}{3E424D}
    \definecolor{ansi-black-intense}{HTML}{282C36}
    \definecolor{ansi-red}{HTML}{E75C58}
    \definecolor{ansi-red-intense}{HTML}{B22B31}
    \definecolor{ansi-green}{HTML}{00A250}
    \definecolor{ansi-green-intense}{HTML}{007427}
    \definecolor{ansi-yellow}{HTML}{DDB62B}
    \definecolor{ansi-yellow-intense}{HTML}{B27D12}
    \definecolor{ansi-blue}{HTML}{208FFB}
    \definecolor{ansi-blue-intense}{HTML}{0065CA}
    \definecolor{ansi-magenta}{HTML}{D160C4}
    \definecolor{ansi-magenta-intense}{HTML}{A03196}
    \definecolor{ansi-cyan}{HTML}{60C6C8}
    \definecolor{ansi-cyan-intense}{HTML}{258F8F}
    \definecolor{ansi-white}{HTML}{C5C1B4}
    \definecolor{ansi-white-intense}{HTML}{A1A6B2}

    % commands and environments needed by pandoc snippets
    % extracted from the output of `pandoc -s`
    \providecommand{\tightlist}{%
      \setlength{\itemsep}{0pt}\setlength{\parskip}{0pt}}
    \DefineVerbatimEnvironment{Highlighting}{Verbatim}{commandchars=\\\{\}}
    % Add ',fontsize=\small' for more characters per line
    \newenvironment{Shaded}{}{}
    \newcommand{\KeywordTok}[1]{\textcolor[rgb]{0.00,0.44,0.13}{\textbf{{#1}}}}
    \newcommand{\DataTypeTok}[1]{\textcolor[rgb]{0.56,0.13,0.00}{{#1}}}
    \newcommand{\DecValTok}[1]{\textcolor[rgb]{0.25,0.63,0.44}{{#1}}}
    \newcommand{\BaseNTok}[1]{\textcolor[rgb]{0.25,0.63,0.44}{{#1}}}
    \newcommand{\FloatTok}[1]{\textcolor[rgb]{0.25,0.63,0.44}{{#1}}}
    \newcommand{\CharTok}[1]{\textcolor[rgb]{0.25,0.44,0.63}{{#1}}}
    \newcommand{\StringTok}[1]{\textcolor[rgb]{0.25,0.44,0.63}{{#1}}}
    \newcommand{\CommentTok}[1]{\textcolor[rgb]{0.38,0.63,0.69}{\textit{{#1}}}}
    \newcommand{\OtherTok}[1]{\textcolor[rgb]{0.00,0.44,0.13}{{#1}}}
    \newcommand{\AlertTok}[1]{\textcolor[rgb]{1.00,0.00,0.00}{\textbf{{#1}}}}
    \newcommand{\FunctionTok}[1]{\textcolor[rgb]{0.02,0.16,0.49}{{#1}}}
    \newcommand{\RegionMarkerTok}[1]{{#1}}
    \newcommand{\ErrorTok}[1]{\textcolor[rgb]{1.00,0.00,0.00}{\textbf{{#1}}}}
    \newcommand{\NormalTok}[1]{{#1}}
    
    % Additional commands for more recent versions of Pandoc
    \newcommand{\ConstantTok}[1]{\textcolor[rgb]{0.53,0.00,0.00}{{#1}}}
    \newcommand{\SpecialCharTok}[1]{\textcolor[rgb]{0.25,0.44,0.63}{{#1}}}
    \newcommand{\VerbatimStringTok}[1]{\textcolor[rgb]{0.25,0.44,0.63}{{#1}}}
    \newcommand{\SpecialStringTok}[1]{\textcolor[rgb]{0.73,0.40,0.53}{{#1}}}
    \newcommand{\ImportTok}[1]{{#1}}
    \newcommand{\DocumentationTok}[1]{\textcolor[rgb]{0.73,0.13,0.13}{\textit{{#1}}}}
    \newcommand{\AnnotationTok}[1]{\textcolor[rgb]{0.38,0.63,0.69}{\textbf{\textit{{#1}}}}}
    \newcommand{\CommentVarTok}[1]{\textcolor[rgb]{0.38,0.63,0.69}{\textbf{\textit{{#1}}}}}
    \newcommand{\VariableTok}[1]{\textcolor[rgb]{0.10,0.09,0.49}{{#1}}}
    \newcommand{\ControlFlowTok}[1]{\textcolor[rgb]{0.00,0.44,0.13}{\textbf{{#1}}}}
    \newcommand{\OperatorTok}[1]{\textcolor[rgb]{0.40,0.40,0.40}{{#1}}}
    \newcommand{\BuiltInTok}[1]{{#1}}
    \newcommand{\ExtensionTok}[1]{{#1}}
    \newcommand{\PreprocessorTok}[1]{\textcolor[rgb]{0.74,0.48,0.00}{{#1}}}
    \newcommand{\AttributeTok}[1]{\textcolor[rgb]{0.49,0.56,0.16}{{#1}}}
    \newcommand{\InformationTok}[1]{\textcolor[rgb]{0.38,0.63,0.69}{\textbf{\textit{{#1}}}}}
    \newcommand{\WarningTok}[1]{\textcolor[rgb]{0.38,0.63,0.69}{\textbf{\textit{{#1}}}}}
    
    
    % Define a nice break command that doesn't care if a line doesn't already
    % exist.
    \def\br{\hspace*{\fill} \\* }
    % Math Jax compatability definitions
    \def\gt{>}
    \def\lt{<}
    % Document parameters
    \title{writeup}
    
    
    

    % Pygments definitions
    
\makeatletter
\def\PY@reset{\let\PY@it=\relax \let\PY@bf=\relax%
    \let\PY@ul=\relax \let\PY@tc=\relax%
    \let\PY@bc=\relax \let\PY@ff=\relax}
\def\PY@tok#1{\csname PY@tok@#1\endcsname}
\def\PY@toks#1+{\ifx\relax#1\empty\else%
    \PY@tok{#1}\expandafter\PY@toks\fi}
\def\PY@do#1{\PY@bc{\PY@tc{\PY@ul{%
    \PY@it{\PY@bf{\PY@ff{#1}}}}}}}
\def\PY#1#2{\PY@reset\PY@toks#1+\relax+\PY@do{#2}}

\expandafter\def\csname PY@tok@w\endcsname{\def\PY@tc##1{\textcolor[rgb]{0.73,0.73,0.73}{##1}}}
\expandafter\def\csname PY@tok@c\endcsname{\let\PY@it=\textit\def\PY@tc##1{\textcolor[rgb]{0.25,0.50,0.50}{##1}}}
\expandafter\def\csname PY@tok@cp\endcsname{\def\PY@tc##1{\textcolor[rgb]{0.74,0.48,0.00}{##1}}}
\expandafter\def\csname PY@tok@k\endcsname{\let\PY@bf=\textbf\def\PY@tc##1{\textcolor[rgb]{0.00,0.50,0.00}{##1}}}
\expandafter\def\csname PY@tok@kp\endcsname{\def\PY@tc##1{\textcolor[rgb]{0.00,0.50,0.00}{##1}}}
\expandafter\def\csname PY@tok@kt\endcsname{\def\PY@tc##1{\textcolor[rgb]{0.69,0.00,0.25}{##1}}}
\expandafter\def\csname PY@tok@o\endcsname{\def\PY@tc##1{\textcolor[rgb]{0.40,0.40,0.40}{##1}}}
\expandafter\def\csname PY@tok@ow\endcsname{\let\PY@bf=\textbf\def\PY@tc##1{\textcolor[rgb]{0.67,0.13,1.00}{##1}}}
\expandafter\def\csname PY@tok@nb\endcsname{\def\PY@tc##1{\textcolor[rgb]{0.00,0.50,0.00}{##1}}}
\expandafter\def\csname PY@tok@nf\endcsname{\def\PY@tc##1{\textcolor[rgb]{0.00,0.00,1.00}{##1}}}
\expandafter\def\csname PY@tok@nc\endcsname{\let\PY@bf=\textbf\def\PY@tc##1{\textcolor[rgb]{0.00,0.00,1.00}{##1}}}
\expandafter\def\csname PY@tok@nn\endcsname{\let\PY@bf=\textbf\def\PY@tc##1{\textcolor[rgb]{0.00,0.00,1.00}{##1}}}
\expandafter\def\csname PY@tok@ne\endcsname{\let\PY@bf=\textbf\def\PY@tc##1{\textcolor[rgb]{0.82,0.25,0.23}{##1}}}
\expandafter\def\csname PY@tok@nv\endcsname{\def\PY@tc##1{\textcolor[rgb]{0.10,0.09,0.49}{##1}}}
\expandafter\def\csname PY@tok@no\endcsname{\def\PY@tc##1{\textcolor[rgb]{0.53,0.00,0.00}{##1}}}
\expandafter\def\csname PY@tok@nl\endcsname{\def\PY@tc##1{\textcolor[rgb]{0.63,0.63,0.00}{##1}}}
\expandafter\def\csname PY@tok@ni\endcsname{\let\PY@bf=\textbf\def\PY@tc##1{\textcolor[rgb]{0.60,0.60,0.60}{##1}}}
\expandafter\def\csname PY@tok@na\endcsname{\def\PY@tc##1{\textcolor[rgb]{0.49,0.56,0.16}{##1}}}
\expandafter\def\csname PY@tok@nt\endcsname{\let\PY@bf=\textbf\def\PY@tc##1{\textcolor[rgb]{0.00,0.50,0.00}{##1}}}
\expandafter\def\csname PY@tok@nd\endcsname{\def\PY@tc##1{\textcolor[rgb]{0.67,0.13,1.00}{##1}}}
\expandafter\def\csname PY@tok@s\endcsname{\def\PY@tc##1{\textcolor[rgb]{0.73,0.13,0.13}{##1}}}
\expandafter\def\csname PY@tok@sd\endcsname{\let\PY@it=\textit\def\PY@tc##1{\textcolor[rgb]{0.73,0.13,0.13}{##1}}}
\expandafter\def\csname PY@tok@si\endcsname{\let\PY@bf=\textbf\def\PY@tc##1{\textcolor[rgb]{0.73,0.40,0.53}{##1}}}
\expandafter\def\csname PY@tok@se\endcsname{\let\PY@bf=\textbf\def\PY@tc##1{\textcolor[rgb]{0.73,0.40,0.13}{##1}}}
\expandafter\def\csname PY@tok@sr\endcsname{\def\PY@tc##1{\textcolor[rgb]{0.73,0.40,0.53}{##1}}}
\expandafter\def\csname PY@tok@ss\endcsname{\def\PY@tc##1{\textcolor[rgb]{0.10,0.09,0.49}{##1}}}
\expandafter\def\csname PY@tok@sx\endcsname{\def\PY@tc##1{\textcolor[rgb]{0.00,0.50,0.00}{##1}}}
\expandafter\def\csname PY@tok@m\endcsname{\def\PY@tc##1{\textcolor[rgb]{0.40,0.40,0.40}{##1}}}
\expandafter\def\csname PY@tok@gh\endcsname{\let\PY@bf=\textbf\def\PY@tc##1{\textcolor[rgb]{0.00,0.00,0.50}{##1}}}
\expandafter\def\csname PY@tok@gu\endcsname{\let\PY@bf=\textbf\def\PY@tc##1{\textcolor[rgb]{0.50,0.00,0.50}{##1}}}
\expandafter\def\csname PY@tok@gd\endcsname{\def\PY@tc##1{\textcolor[rgb]{0.63,0.00,0.00}{##1}}}
\expandafter\def\csname PY@tok@gi\endcsname{\def\PY@tc##1{\textcolor[rgb]{0.00,0.63,0.00}{##1}}}
\expandafter\def\csname PY@tok@gr\endcsname{\def\PY@tc##1{\textcolor[rgb]{1.00,0.00,0.00}{##1}}}
\expandafter\def\csname PY@tok@ge\endcsname{\let\PY@it=\textit}
\expandafter\def\csname PY@tok@gs\endcsname{\let\PY@bf=\textbf}
\expandafter\def\csname PY@tok@gp\endcsname{\let\PY@bf=\textbf\def\PY@tc##1{\textcolor[rgb]{0.00,0.00,0.50}{##1}}}
\expandafter\def\csname PY@tok@go\endcsname{\def\PY@tc##1{\textcolor[rgb]{0.53,0.53,0.53}{##1}}}
\expandafter\def\csname PY@tok@gt\endcsname{\def\PY@tc##1{\textcolor[rgb]{0.00,0.27,0.87}{##1}}}
\expandafter\def\csname PY@tok@err\endcsname{\def\PY@bc##1{\setlength{\fboxsep}{0pt}\fcolorbox[rgb]{1.00,0.00,0.00}{1,1,1}{\strut ##1}}}
\expandafter\def\csname PY@tok@kc\endcsname{\let\PY@bf=\textbf\def\PY@tc##1{\textcolor[rgb]{0.00,0.50,0.00}{##1}}}
\expandafter\def\csname PY@tok@kd\endcsname{\let\PY@bf=\textbf\def\PY@tc##1{\textcolor[rgb]{0.00,0.50,0.00}{##1}}}
\expandafter\def\csname PY@tok@kn\endcsname{\let\PY@bf=\textbf\def\PY@tc##1{\textcolor[rgb]{0.00,0.50,0.00}{##1}}}
\expandafter\def\csname PY@tok@kr\endcsname{\let\PY@bf=\textbf\def\PY@tc##1{\textcolor[rgb]{0.00,0.50,0.00}{##1}}}
\expandafter\def\csname PY@tok@bp\endcsname{\def\PY@tc##1{\textcolor[rgb]{0.00,0.50,0.00}{##1}}}
\expandafter\def\csname PY@tok@fm\endcsname{\def\PY@tc##1{\textcolor[rgb]{0.00,0.00,1.00}{##1}}}
\expandafter\def\csname PY@tok@vc\endcsname{\def\PY@tc##1{\textcolor[rgb]{0.10,0.09,0.49}{##1}}}
\expandafter\def\csname PY@tok@vg\endcsname{\def\PY@tc##1{\textcolor[rgb]{0.10,0.09,0.49}{##1}}}
\expandafter\def\csname PY@tok@vi\endcsname{\def\PY@tc##1{\textcolor[rgb]{0.10,0.09,0.49}{##1}}}
\expandafter\def\csname PY@tok@vm\endcsname{\def\PY@tc##1{\textcolor[rgb]{0.10,0.09,0.49}{##1}}}
\expandafter\def\csname PY@tok@sa\endcsname{\def\PY@tc##1{\textcolor[rgb]{0.73,0.13,0.13}{##1}}}
\expandafter\def\csname PY@tok@sb\endcsname{\def\PY@tc##1{\textcolor[rgb]{0.73,0.13,0.13}{##1}}}
\expandafter\def\csname PY@tok@sc\endcsname{\def\PY@tc##1{\textcolor[rgb]{0.73,0.13,0.13}{##1}}}
\expandafter\def\csname PY@tok@dl\endcsname{\def\PY@tc##1{\textcolor[rgb]{0.73,0.13,0.13}{##1}}}
\expandafter\def\csname PY@tok@s2\endcsname{\def\PY@tc##1{\textcolor[rgb]{0.73,0.13,0.13}{##1}}}
\expandafter\def\csname PY@tok@sh\endcsname{\def\PY@tc##1{\textcolor[rgb]{0.73,0.13,0.13}{##1}}}
\expandafter\def\csname PY@tok@s1\endcsname{\def\PY@tc##1{\textcolor[rgb]{0.73,0.13,0.13}{##1}}}
\expandafter\def\csname PY@tok@mb\endcsname{\def\PY@tc##1{\textcolor[rgb]{0.40,0.40,0.40}{##1}}}
\expandafter\def\csname PY@tok@mf\endcsname{\def\PY@tc##1{\textcolor[rgb]{0.40,0.40,0.40}{##1}}}
\expandafter\def\csname PY@tok@mh\endcsname{\def\PY@tc##1{\textcolor[rgb]{0.40,0.40,0.40}{##1}}}
\expandafter\def\csname PY@tok@mi\endcsname{\def\PY@tc##1{\textcolor[rgb]{0.40,0.40,0.40}{##1}}}
\expandafter\def\csname PY@tok@il\endcsname{\def\PY@tc##1{\textcolor[rgb]{0.40,0.40,0.40}{##1}}}
\expandafter\def\csname PY@tok@mo\endcsname{\def\PY@tc##1{\textcolor[rgb]{0.40,0.40,0.40}{##1}}}
\expandafter\def\csname PY@tok@ch\endcsname{\let\PY@it=\textit\def\PY@tc##1{\textcolor[rgb]{0.25,0.50,0.50}{##1}}}
\expandafter\def\csname PY@tok@cm\endcsname{\let\PY@it=\textit\def\PY@tc##1{\textcolor[rgb]{0.25,0.50,0.50}{##1}}}
\expandafter\def\csname PY@tok@cpf\endcsname{\let\PY@it=\textit\def\PY@tc##1{\textcolor[rgb]{0.25,0.50,0.50}{##1}}}
\expandafter\def\csname PY@tok@c1\endcsname{\let\PY@it=\textit\def\PY@tc##1{\textcolor[rgb]{0.25,0.50,0.50}{##1}}}
\expandafter\def\csname PY@tok@cs\endcsname{\let\PY@it=\textit\def\PY@tc##1{\textcolor[rgb]{0.25,0.50,0.50}{##1}}}

\def\PYZbs{\char`\\}
\def\PYZus{\char`\_}
\def\PYZob{\char`\{}
\def\PYZcb{\char`\}}
\def\PYZca{\char`\^}
\def\PYZam{\char`\&}
\def\PYZlt{\char`\<}
\def\PYZgt{\char`\>}
\def\PYZsh{\char`\#}
\def\PYZpc{\char`\%}
\def\PYZdl{\char`\$}
\def\PYZhy{\char`\-}
\def\PYZsq{\char`\'}
\def\PYZdq{\char`\"}
\def\PYZti{\char`\~}
% for compatibility with earlier versions
\def\PYZat{@}
\def\PYZlb{[}
\def\PYZrb{]}
\makeatother


    % Exact colors from NB
    \definecolor{incolor}{rgb}{0.0, 0.0, 0.5}
    \definecolor{outcolor}{rgb}{0.545, 0.0, 0.0}



    
    % Prevent overflowing lines due to hard-to-break entities
    \sloppy 
    % Setup hyperref package
    \hypersetup{
      breaklinks=true,  % so long urls are correctly broken across lines
      colorlinks=true,
      urlcolor=urlcolor,
      linkcolor=linkcolor,
      citecolor=citecolor,
      }
    % Slightly bigger margins than the latex defaults
    
    \geometry{verbose,tmargin=1in,bmargin=1in,lmargin=1in,rmargin=1in}
    
    

    \begin{document}
    
    
    \maketitle
    
    

    
    \subsection{Writeup Template}\label{writeup-template}

\subsubsection{You can use this file as a template for your writeup if
you want to submit it as a markdown file, but feel free to use some
other method and submit a pdf if you
prefer.}\label{you-can-use-this-file-as-a-template-for-your-writeup-if-you-want-to-submit-it-as-a-markdown-file-but-feel-free-to-use-some-other-method-and-submit-a-pdf-if-you-prefer.}

\begin{center}\rule{0.5\linewidth}{\linethickness}\end{center}

\textbf{Advanced Lane Finding Project}

The goals / steps of this project are the following:

\begin{itemize}
\tightlist
\item
  Compute the camera calibration matrix and distortion coefficients
  given a set of chessboard images.
\item
  Apply a distortion correction to raw images.
\item
  Use color transforms, gradients, etc., to create a thresholded binary
  image.
\item
  Apply a perspective transform to rectify binary image ("birds-eye
  view").
\item
  Detect lane pixels and fit to find the lane boundary.
\item
  Determine the curvature of the lane and vehicle position with respect
  to center.
\item
  Warp the detected lane boundaries back onto the original image.
\item
  Output visual display of the lane boundaries and numerical estimation
  of lane curvature and vehicle position.
\end{itemize}

\subsection{\texorpdfstring{\href{https://review.udacity.com/\#!/rubrics/571/view}{Rubric}
Points}{Rubric Points}}\label{rubric-points}

\subsubsection{Here I will consider the rubric points individually and
describe how I addressed each point in my
implementation.}\label{here-i-will-consider-the-rubric-points-individually-and-describe-how-i-addressed-each-point-in-my-implementation.}

\begin{center}\rule{0.5\linewidth}{\linethickness}\end{center}

\subsubsection{Writeup / README}\label{writeup-readme}

\paragraph{\texorpdfstring{1. Provide a Writeup / README that includes
all the rubric points and how you addressed each one. You can submit
your writeup as markdown or pdf.
\href{https://github.com/udacity/CarND-Advanced-Lane-Lines/blob/master/writeup_template.md}{Here}
is a template writeup for this project you can use as a guide and a
starting
point.}{1. Provide a Writeup / README that includes all the rubric points and how you addressed each one. You can submit your writeup as markdown or pdf. Here is a template writeup for this project you can use as a guide and a starting point.}}\label{provide-a-writeup-readme-that-includes-all-the-rubric-points-and-how-you-addressed-each-one.-you-can-submit-your-writeup-as-markdown-or-pdf.-here-is-a-template-writeup-for-this-project-you-can-use-as-a-guide-and-a-starting-point.}

You're reading it!

\subsubsection{Camera Calibration}\label{camera-calibration}

\paragraph{1. Briefly state how you computed the camera matrix and
distortion coefficients. Provide an example of a distortion corrected
calibration
image.}\label{briefly-state-how-you-computed-the-camera-matrix-and-distortion-coefficients.-provide-an-example-of-a-distortion-corrected-calibration-image.}

The code for this step is contained in the 2nd and 3rd cells of the
notebook located in "Advaned\_Lane\_Finding.ipynb".

In \(2nd\) cell, two variables are difined: 'objpoints' and 'imgpoints'.
'objpoints' is defined with the shape of {[}9*6, 3{]} based on the
number of the chessboard corners in the given camera calibration images.
It is assumed that the chessboard is fixed on the (x, y) plane at z=0,
therefore the object points are the same for each calibration image.
Thus, \texttt{objp} is just a replicated array of coordinates, and
\texttt{objpoints} will be appended with a copy of it every time when it
is successfully detected all chessboard corners in a test image.
\texttt{imgpoints} will be appended with the (x, y) pixel position of
each of the corners in the tested image plane with each successful
chessboard detection.

In \(3rd\) cell, the output \texttt{objpoints} and \texttt{imgpoints}
are used to compute the camera calibration and distortion coefficients
using the \texttt{cv2.calibrateCamera()} function. Then this distortion
correction is applied to the test image using the
\texttt{cv2.undistort()} function and obtained this result:

\begin{verbatim}
<img  src="./camera_cal/calibration2.jpg" alt="Drawing" style="width: 350px;"/>
<figcaption>Distorted</figcaption>
</figure></td>
\end{verbatim}

\begin{verbatim}
<img  src="./output_images/calibrated2.jpg" alt="Drawing" style="width: 350px;"/>
<figcaption>Un_Distorted</figcaption></figure></td>
\end{verbatim}

\subsubsection{Pipeline (single images)}\label{pipeline-single-images}

\paragraph{1. Provide an example of a distortion-corrected
image.}\label{provide-an-example-of-a-distortion-corrected-image.}

To demonstrate this step, the same distortion correction is appled to
one of the test images like this one:

\begin{verbatim}
<img  src="./test_images/test6.jpg" alt="Drawing" style="width: 350px;"/>
<figcaption>Distorted</figcaption>
</figure></td>
\end{verbatim}

\begin{verbatim}
<img  src="./output_images/calibrated_test6.jpg" alt="Drawing" style="width: 350px;"/>
<figcaption>Un_Distorted</figcaption></figure></td>
\end{verbatim}

\paragraph{2. Describe how (and identify where in your code) you
performed a perspective transform and provide an example of a
transformed
image.}\label{describe-how-and-identify-where-in-your-code-you-performed-a-perspective-transform-and-provide-an-example-of-a-transformed-image.}

In \(4th\) cell, the code for perspective transform is included: First
the source points and destination points are defined as shown below:

\begin{Shaded}
\begin{Highlighting}[]
\NormalTok{src }\OperatorTok{=}\NormalTok{ np.float32([[}\DecValTok{205}\NormalTok{,}\DecValTok{720}\NormalTok{],[}\DecValTok{565}\NormalTok{,}\DecValTok{470}\NormalTok{],[}\DecValTok{720}\NormalTok{,}\DecValTok{470}\NormalTok{],[}\DecValTok{1100}\NormalTok{,}\DecValTok{720}\NormalTok{]])}
\NormalTok{h, w }\OperatorTok{=}\NormalTok{ img.shape[:}\DecValTok{2}\NormalTok{]}
\NormalTok{dst }\OperatorTok{=}\NormalTok{ np.float32([[}\DecValTok{350}\NormalTok{, }\DecValTok{720}\NormalTok{],[}\DecValTok{350}\NormalTok{, }\DecValTok{0}\NormalTok{],[w}\OperatorTok{-}\DecValTok{350}\NormalTok{, }\DecValTok{0}\NormalTok{],[w}\OperatorTok{-}\DecValTok{350}\NormalTok{, }\DecValTok{720}\NormalTok{]])}
\end{Highlighting}
\end{Shaded}

This resulted in the following source and destination points:

\begin{longtable}[]{@{}cc@{}}
\toprule
Source & Destination\tabularnewline
\midrule
\endhead
205, 720 & 350, 720\tabularnewline
565, 470 & 350, 0\tabularnewline
720, 470 & 930, 0\tabularnewline
1100, 720 & 930, 720\tabularnewline
\bottomrule
\end{longtable}

It is verified that the perspective transform was working as expected by
drawing the \texttt{src} and \texttt{dst} points onto a test image and
its warped counterpart to verify that the lines appear parallel in the
warped image.

\begin{verbatim}
<img  src="./output_images/source_points_drawn.jpg" alt="Drawing" style="width: 350px;"/>
<figcaption>Test image</figcaption>
</figure></td>
\end{verbatim}

\begin{verbatim}
<img  src="./output_images/dest_points_drawn.jpg" alt="Drawing" style="width: 350px;"/>
<figcaption>Warped image</figcaption></figure></td>
\end{verbatim}

\paragraph{3. Describe how (and identify where in your code) you used
color transforms, gradients or other methods to create a thresholded
binary image. Provide an example of a binary image
result.}\label{describe-how-and-identify-where-in-your-code-you-used-color-transforms-gradients-or-other-methods-to-create-a-thresholded-binary-image.-provide-an-example-of-a-binary-image-result.}

The color space been explored include: * RGB: in \(9th\) cell\\
*** Red channel is chosen with threshold of 215/255 * HLS: in \(10th\)
cell\\
*** S channel is chosen with threshold of 210/255 * HSV: in \(11th\)
cell\\
*** V channel is chosen with threshold of 220/255 * The code for all
three above color space are of the same template, therefore the code for
LUV is ignored\\
*** V channel in LUV is chosen with threshold of 155/255

\begin{itemize}
\tightlist
\item
  X Gradient is chosen with threshold of 14/100
\end{itemize}

The following table \textbf{\(\checkmark\)} shows the channels which
generate highest lane information

\begin{center}\rule{0.5\linewidth}{\linethickness}\end{center}

\paragraph{Observations of RGB color
space:}\label{observations-of-rgb-color-space}

\begin{longtable}[]{@{}llll@{}}
\toprule
images & Red(215,255) & Green(195,255) & Blue(60,210)\tabularnewline
\midrule
\endhead
straigh\_line1.jpg & \textbf{\(\checkmark\)} & \textbf{\(\checkmark\)}
&\tabularnewline
straigh\_line2.jpg & \textbf{\(\checkmark\)} & \textbf{\(\checkmark\)}
&\tabularnewline
test1.jpg & \textbf{\(\checkmark\)} & &\tabularnewline
test2.jpg & \textbf{\(\checkmark\)} & \textbf{\(\checkmark\)}
&\tabularnewline
test3.jpg & \textbf{\(\checkmark\)} & \textbf{\(\checkmark\)}
&\tabularnewline
test4.jpg & \textbf{\(\checkmark\)} & &\tabularnewline
test5.jpg & & &\tabularnewline
test6.jpg & \textbf{\(\checkmark\)} & &\tabularnewline
\bottomrule
\end{longtable}

\begin{verbatim}
<figcaption>Warped Image</figcaption>
<img  src="./output_images/warped_straight_lines1.jpg" alt="Drawing" style="width: 350px;"/>
</figure></td>
\end{verbatim}

\begin{verbatim}
<figcaption>Lanes info in Red channel</figcaption>
<img  src="./output_images/straight_lines1R_RGB.jpg" alt="Drawing" style="width: 350px;"/>
\end{verbatim}

\begin{center}\rule{0.5\linewidth}{\linethickness}\end{center}

\paragraph{Observations of HLS color
space:}\label{observations-of-hls-color-space}

\begin{longtable}[]{@{}llll@{}}
\toprule
images & H(14,85) & L(120,255) & S(210,255)\tabularnewline
\midrule
\endhead
straigh\_line1.jpg & & \textbf{\(\checkmark\)} &\tabularnewline
straigh\_line2.jpg & & \textbf{\(\checkmark\)} &\tabularnewline
test1.jpg & & & \textbf{\(\checkmark\)}\tabularnewline
test2.jpg & & \textbf{\(\checkmark\)} &\tabularnewline
test3.jpg & & \textbf{\(\checkmark\)} &\tabularnewline
test4.jpg & & & \textbf{\(\checkmark\)}\tabularnewline
test5.jpg & & &\tabularnewline
test6.jpg & & \textbf{\(\checkmark\)} &\tabularnewline
\bottomrule
\end{longtable}

\begin{verbatim}
<figcaption>Warped Image</figcaption>
<img  src="./output_images/warped_straight_lines1.jpg" alt="Drawing" style="width: 350px;"/>
</figure></td>
\end{verbatim}

\begin{verbatim}
<figcaption>Lanes info in S/HLS channel</figcaption>
<img  src="./output_images/straight_lines1S_HLS.jpg" alt="Drawing" style="width: 350px;"/>
\end{verbatim}

\begin{center}\rule{0.5\linewidth}{\linethickness}\end{center}

\paragraph{Observations of HSV color
space:}\label{observations-of-hsv-color-space}

\begin{longtable}[]{@{}llll@{}}
\toprule
images & H(14,85) & S(80,255) & V(220,255)\tabularnewline
\midrule
\endhead
straigh\_line1.jpg & & & \textbf{\(\checkmark\)}\tabularnewline
straigh\_line2.jpg & & & \textbf{\(\checkmark\)}\tabularnewline
test1.jpg & & & \textbf{\(\checkmark\)}\tabularnewline
test2.jpg & & & \textbf{\(\checkmark\)}\tabularnewline
test3.jpg & & & \textbf{\(\checkmark\)}\tabularnewline
test4.jpg & & & \textbf{\(\checkmark\)}\tabularnewline
test5.jpg & & &\tabularnewline
test6.jpg & & & \textbf{\(\checkmark\)}\tabularnewline
\bottomrule
\end{longtable}

\begin{verbatim}
<figcaption>Warped Image</figcaption>
<img  src="./output_images/warped_straight_lines1.jpg" alt="Drawing" style="width: 350px;"/>
</figure></td>
\end{verbatim}

\begin{verbatim}
<figcaption>Lanes info in V/HSV channel</figcaption>
<img  src="./output_images/straight_lines1V_HSV.jpg" alt="Drawing" style="width: 350px;"/>
\end{verbatim}

\begin{center}\rule{0.5\linewidth}{\linethickness}\end{center}

\paragraph{Observations of gradient:}\label{observations-of-gradient}

\begin{longtable}[]{@{}llll@{}}
\toprule
images & X(14,100) & Y(0,255) & ABS(0,255)\tabularnewline
\midrule
\endhead
straigh\_line1.jpg & \textbf{\(\checkmark\)} & &\tabularnewline
straigh\_line2.jpg & \textbf{\(\checkmark\)} & &\tabularnewline
test1.jpg & & &\tabularnewline
test2.jpg & \textbf{\(\checkmark\)} & &\tabularnewline
test3.jpg & \textbf{\(\checkmark\)} & &\tabularnewline
test4.jpg & & &\tabularnewline
test5.jpg & \textbf{\(\checkmark\)} & &\tabularnewline
test6.jpg & \textbf{\(\checkmark\)} & &\tabularnewline
\bottomrule
\end{longtable}

\begin{verbatim}
<figcaption>Warped Image</figcaption>
<img  src="./output_images/warped_straight_lines1.jpg" alt="Drawing" style="width: 350px;"/>
</figure></td>
\end{verbatim}

\begin{verbatim}
<figcaption>Lanes info in X_gradient channel</figcaption>
<img  src="./output_images/straight_lines1X_Gradient.jpg" alt="Drawing" style="width: 350px;"/>
\end{verbatim}

\subsubsection{The final combined image
is:}\label{the-final-combined-image-is}

The final combined image is the sume of all above images, then set pixel
value \textless{}=1 as 0

\begin{verbatim}
<figcaption>            Lanes info in final combined image</figcaption>
<img  src="./output_images/straight_lines1_binary.jpg" alt="Drawing" style="width: 350px;"/>
\end{verbatim}

\paragraph{4. Describe how (and identify where in your code) you
identified lane-line pixels and fit their positions with a
polynomial?}\label{describe-how-and-identify-where-in-your-code-you-identified-lane-line-pixels-and-fit-their-positions-with-a-polynomial}

This step is done in cell \(14\)\\
Function 'find\_lane\_pixels(binary\_warped)' is to find pixels
belonging to left lane and right lane.

The histogram of the first video frame is calculated, and the two
highest points of the histogram are taken as the lane starting points.

Two sliding windows are defined for both left and right lanes and been
moved upward to search all potential lane pixels.

The a 2nd-order polynomial lines are calculated to fit those lane
pixels.

\begin{verbatim}
<figcaption>Original Image</figcaption>
<img  src="./output_images/warped_test6.jpg" alt="Drawing" style="width: 350px;"/>
</figure></td>
\end{verbatim}

\begin{verbatim}
<figcaption>Curve_fit image</figcaption>
<img  src="./output_images/curve_fit.png" alt="Drawing" style="width: 350px;"/>
\end{verbatim}

\paragraph{5. Describe how (and identify where in your code) you
calculated the radius of curvature of the lane and the position of the
vehicle with respect to
center.}\label{describe-how-and-identify-where-in-your-code-you-calculated-the-radius-of-curvature-of-the-lane-and-the-position-of-the-vehicle-with-respect-to-center.}

The curvatures are calculated in cell \(14\) First define meters per pix
in both x and y direction

\begin{Shaded}
\begin{Highlighting}[]
\NormalTok{my_per_pix }\OperatorTok{=} \DecValTok{30}\OperatorTok{/} \DecValTok{720}
\NormalTok{mx_per_pix }\OperatorTok{=} \FloatTok{3.7}\OperatorTok{/} \DecValTok{700}
\end{Highlighting}
\end{Shaded}

Then find the 2nd order polynominal curve fitting for both left and
right lanes

\begin{Shaded}
\begin{Highlighting}[]
\NormalTok{left_fit }\OperatorTok{=}\NormalTok{ np.polyfit(lefty, leftx, }\DecValTok{2}\NormalTok{)}
\NormalTok{right_fit }\OperatorTok{=}\NormalTok{ np.polyfit(righty, rightx, }\DecValTok{2}\NormalTok{)}
\end{Highlighting}
\end{Shaded}

Then conver those coefficients 'left\_fit' and 'right\_fit', which have
unit in pixels, to unit in 'meter'

\begin{Shaded}
\begin{Highlighting}[]
\NormalTok{left_fit_cr_0 }\OperatorTok{=}\NormalTok{ mx_per_pix }\OperatorTok{/}\NormalTok{ my_per_pix}\OperatorTok{**}\DecValTok{2} \OperatorTok{*}\NormalTok{ left_fit[}\DecValTok{0}\NormalTok{]}
\NormalTok{left_fit_cr_1 }\OperatorTok{=}\NormalTok{ mx_per_pix }\OperatorTok{/}\NormalTok{ my_per_pix }\OperatorTok{*}\NormalTok{ left_fit[}\DecValTok{1}\NormalTok{]}
\NormalTok{right_fit_cr_0 }\OperatorTok{=}\NormalTok{ mx_per_pix }\OperatorTok{/}\NormalTok{ my_per_pix}\OperatorTok{**}\DecValTok{2} \OperatorTok{*}\NormalTok{ right_fit[}\DecValTok{0}\NormalTok{]}
\NormalTok{right_fit_cr_1 }\OperatorTok{=}\NormalTok{ mx_per_pix }\OperatorTok{/}\NormalTok{ my_per_pix }\OperatorTok{*}\NormalTok{ right_fit[}\DecValTok{1}\NormalTok{]    }
\end{Highlighting}
\end{Shaded}

The curvature is calculated as:

\begin{Shaded}
\begin{Highlighting}[]
\NormalTok{left_curverad }\OperatorTok{=}\NormalTok{ ((}\DecValTok{1} \OperatorTok{+}\NormalTok{ (}\DecValTok{2}\OperatorTok{*}\NormalTok{left_fit_cr_0}\OperatorTok{*}\NormalTok{y_eval}\OperatorTok{*}\NormalTok{my_per_pix }\OperatorTok{+}\NormalTok{ left_fit_cr_1)}\OperatorTok{**}\DecValTok{2}\NormalTok{)}\OperatorTok{**}\FloatTok{1.5}\NormalTok{) }\OperatorTok{/}\NormalTok{ np.absolute(}\DecValTok{2}\OperatorTok{*}\NormalTok{left_fit_cr_0)}
\NormalTok{right_curverad }\OperatorTok{=}\NormalTok{ ((}\DecValTok{1} \OperatorTok{+}\NormalTok{ (}\DecValTok{2}\OperatorTok{*}\NormalTok{right_fit_cr_0}\OperatorTok{*}\NormalTok{y_eval}\OperatorTok{*}\NormalTok{my_per_pix }\OperatorTok{+}\NormalTok{ right_fit_cr_1)}\OperatorTok{**}\DecValTok{2}\NormalTok{)}\OperatorTok{**}\FloatTok{1.5}\NormalTok{) }\OperatorTok{/}\NormalTok{ np.absolute(}\DecValTok{2}\OperatorTok{*}\NormalTok{right_fit_cr_0)}
\end{Highlighting}
\end{Shaded}

based on the following equation:

\begin{equation*}
R_{curve} =\frac{(1+(2Ay+B)^2)^{\frac32})}{|2A|}
\end{equation*}

\paragraph{6. Provide an example image of your result plotted back down
onto the road such that the lane area is identified
clearly.}\label{provide-an-example-image-of-your-result-plotted-back-down-onto-the-road-such-that-the-lane-area-is-identified-clearly.}

I implemented this step in cell \(15\). Here is an example of my result
on a test image:

\begin{verbatim}
<figcaption>Image with lane area</figcaption>
<img  src="./output_images/lane_area.png" alt="Drawing" style="width: 350px;"/>
\end{verbatim}

\begin{center}\rule{0.5\linewidth}{\linethickness}\end{center}

\subsubsection{Pipeline (video)}\label{pipeline-video}

\paragraph{1. Provide a link to your final video output. Your pipeline
should perform reasonably well on the entire project video (wobbly lines
are ok but no catastrophic failures that would cause the car to drive
off the
road!).}\label{provide-a-link-to-your-final-video-output.-your-pipeline-should-perform-reasonably-well-on-the-entire-project-video-wobbly-lines-are-ok-but-no-catastrophic-failures-that-would-cause-the-car-to-drive-off-the-road.}

Here's a \href{./output_videos/final_video.mp4}{link to my video result}

\begin{center}\rule{0.5\linewidth}{\linethickness}\end{center}

\subsubsection{Discussion}\label{discussion}

\paragraph{1. Briefly discuss any problems / issues you faced in your
implementation of this project. Where will your pipeline likely fail?
What could you do to make it more
robust?}\label{briefly-discuss-any-problems-issues-you-faced-in-your-implementation-of-this-project.-where-will-your-pipeline-likely-fail-what-could-you-do-to-make-it-more-robust}

If there is no clear histogram peaks in the first image frame, it will
be hard to find these two starting points\\
If there is no clear starting points for those frames in the middle of
video, the ending point of the past frame can be saved and used as the
starting point of future frames\\
If there is no clear starting points for \(several\) \(consecutive\)
frames, the lane pixels of past several frams can be save and a
2nd-order polynomial fitting can be made to predict the lane pixels for
the future frames\\
In this project, only X direction gradient is used. If there is a sharp
turn, the combined (of x and y direction) could be better


    % Add a bibliography block to the postdoc
    
    
    
    \end{document}
